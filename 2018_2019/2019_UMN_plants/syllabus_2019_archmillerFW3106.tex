\documentclass{tufte-handout}
\usepackage{fontspec}
\usepackage{termcal}

\makeatletter
\providecommand\tuftedate{}
\@ifpackageloaded{termcal}{%
  \renewcommand{\date}[1]{%
    \gdef\@date{#1}%
    \begingroup%
    % TODO store contents of \thanks command
    \renewcommand{\thanks}[1]{}% swallow \thanks contents
    \protected@xdef\tuftedate{#1}%
    \endgroup%
  }{%
    % Do nothing else, there's no need to redefine \date
  }
}
\makeatother
\defaultfontfeatures{Mapping=tex-text}

\renewcommand{\allcapsspacing}[1]{{\addfontfeature{LetterSpace=20.0}#1}}
\renewcommand{\smallcapsspacing}[1]{{\addfontfeature{LetterSpace=5.0}#1}}
\renewcommand{\textsc}[1]{\smallcapsspacing{\textsmallcaps{#1}}}
\renewcommand{\smallcaps}[1]{\smallcapsspacing{\scshape\MakeTextLowercase{#1}}}

\renewcommand{\calprintclass}{}


\title{Plant Identification \& Vegetation Sampling for Habitat Assessments}										% change each year
\author{FW 3106}										% change per section
\date{2019 August Field Session: 1 Credit}

\begin{document}



\maketitle


\newthought{Lead Co-Instructor: Althea A.~ArchMiller\marginnote{The schedules and policies associated with this course may be subject to revision or change as a consequence of changing circumstances or events. Reasonable notification will be provided to students prior to any major changes in course policies or procedures.}}\\
218.556.8053 (cell)\\
Email: aarchmil@cord.edu\\
Twitter: @aaarchmiller\\


\newthought{Co-Instructor: Michael Lynch}\\
651.208.8734 (cell)\\
Email: mikassa.lynch@gmail.com\\

\begin{fullwidth}

\section{Course Description}

In this course, students will be introduced to common vegetation sampling methods used for habitat assessments. Students will learn to identify up to 110 plant species typical of Minnesota’s forest, prairie, wetland, and lacustrine ecosystems using field guides, taxonomic keys, and readily observable traits. Students will also learn about the importance of plant species composition, diversity, and structure in providing habitat for wildlife, as well as the role of ecological restoration and management in maintaining habitat quality. Coursework will include plant identification hikes, vegetation sampling exercises, participation and professionalism, one or more written assignments, plant identification quizzes, one vegetation sampling practical exam, and one plant identification final exam.

\subsection{Learning Outcomes}

Over the next three weeks, students will:

\begin{enumerate}
	\item Become familiar with plant traits commonly used to identify native plants, including leaf and stem morphology, growth form and structure, bark characteristics, and floral and fruit morphology.
	\item Learn to identify and recognize up to 110 common native and introduced plant species based on readily observable traits and habitat.
	\item Learn to use floristic keys and plant guides to correctly identify unknown plants and distinguish between similarly appearing species.
	\item Learn how to characterize vegetation using standard sampling methods, and compare the effectiveness of different methods.
	\item Understand basic native plant community concepts as they relate to fish and wildlife habitats.
	\item Understand the ways in which modern ecological land classifications integrate vegetation types to the larger geomorphic and climatic context where organisms inhabit.
\end{enumerate}

\subsection{Website}

Course materials can be accessed via the FW3108 Canvas website under the ``Plant Sampling Resources'' link at the bottom of the home page. Check the FW3108 Canvas site frequently to make sure you are current with FW3106 postings and resources.

\subsection{Communication}

The course instructors will use the regularly scheduled plant field sessions to communicate with the students regarding course announcements, schedules, and changes or modifications to the schedule or syllabus. Printed material will be handed out to the students at the beginning of these field sessions. Electronic versions of the handouts and additional supporting materials may be accessed on the FW3018 Canvas site under the Plant Sampling Resources link. In addition, the students may contact Althea ArchMiller and Michael Lynch via email or cell phone with additional course questions. Additional opportunities to learn the plant material and other course material will be offered outside of regularly scheduled field sessions in the form of plant walks or study sessions. Students are encouraged to request assistance from Althea ArchMiller and Michael Lynch throughout the August field session.

\subsection{Texts} 

The required textbook for this course is \emph{Key Plants Appearing in the Field Guides to Native Plant Communities of Minnesota: Forests \& Woodlands}, Second Edition (Almendinger 2015), and is available for download as a PDF from the course Canvas site. All students must download this plant identification manual and use it as their reference for reviewing plants taught in the field during the plant identification hikes. There are also four printed copies of this manual available for student use in the plant resource library in the plant lab. Supplemental resources and readings will be made available via the course website and resource library. Plant lists and supporting material will be distributed to all students during the scheduled field sessions.

\newthought{Students are strongly encouraged} to bring at least one additional plant field guide of their choice. Students are encouraged to purchase and reference the publication \emph{The Field Guide to the Native Plant Communities of Minnesota – The Laurentian Mixed Forest Province} (MN DNR, 2003, available at the Minnesota Bookstore: https://mn.gov/admin/bookstore), which now serves as the preferred field reference for native vegetation and plant community classification of the Cloquet and Itasca landscapes. 

Finally, the Minnesota Wildflowers app is free and available for most smart phones. This is a nice way to identify unknown flowering plants and grasses in Minnesota. 

\subsection{Attendance}

Because of the compressed nature of this course, it is important that students attend all lectures and activities. Attendance is required. There is no good way to make up a field experience. The only acceptable reasons for missing any session during the class are: documented illness, documented family emergency, subpoenas, jury duty, military service, bereavement, and religious observances. Each unexcused absence will result in a 5\% deduction from your course total.

This is a 1-credit course, which assumes 60 hours of work, according to University of Minnesota expectations. Students should therefore expect to spend 30 hours outside of class ($\sim$10 hours/week) in addition to the 30 hours of class time in order to master the material and complete assignments.

\subsection{Respect for Diversity}

It is our intent that students from diverse backgrounds and perspectives be well-served by this course, and that the diversity that students bring to this class be viewed as a resource. Please let us know ways to improve the effectiveness of the course for you, personally, or for other students or student groups. As a student in this class, you are required to treat other members of the class with respect and kindness. Disrespectful, rude, or exclusive behavior will not be tolerated.

\section{Grades}

Class sessions will typically include a plant identification hike and/or vegetation sampling exercises. The course instructors will introduce key concepts and terminology via infrequent brief lectures, and occasionally we will re-enforce certain skills via classroom activities, but the majority of our class time will be outdoors. All students will be evaluated based on the following components.

%Final grades will be based on the following:

\begin{table}
\begin{tabular}{l l  r r}
Item & Details & Points & \% \\
\hline
Plant Sampling Activities &  & 100 & 10 \\
Plant Sampling Practical Exam & Aug.\ 19 or 20 & 100 & 10 \\
Plant Identification Quizzes & Up to 7 quizzes & 350 & 35 \\
Final Exam & Aug.\ 24 & 250  & 25 \\						% change for each unit
Community Narrative Paper & Due Aug.\ 30 & 100 & 10 \\
Professionalism & & 100 & 10 \\
%& Homework & varying dates & 5 \\
\hline
& & Total 1000 & 100
\end{tabular}
\end{table}

\end{fullwidth}

\textbf{Plant Sampling Exercises:} Students will work in pairs or groups of 6 to complete 2 to 3 sampling activities in class during the three week field session, including an aquatic plant survey at Lake Ozawindib. Students will turn in completed data sheets at the end of each exercise (unless otherwise specified). Vegetation sampling is an important comprehensive skill to understand while completing the FW3106 coursework. 

\marginnote{Late assignments will not be accepted and missed quizzes cannot be made up. There are no extra-credit opportunities in this class.}

\textbf{Plant Sampling Practical Exam:} Students will be tested in their ability to use plant sampling techniques as taught by the instructors. 

\begin{margintable}
\begin{tabular}{rl}
Percentage & Grade \\
\hline 
$\ge93$ & A \\
90-92.9 & A- \\
87-89.9 & B+ \\
83-86.9 & B \\
80-82.9 & B- \\
77-79.9 & C+ \\
73-76.9 & C \\
70-72.9 & C- \\
67-69.9 & D+ \\
60-66.9 & D \\
$<60$ & F \\
\hline
\end{tabular}
\end{margintable}

	

\textbf{Plant Identification Quizzes} will be given at the beginning of most class sessions, other than the first, during which we will ask students to identify 10 species taught in the previous field session(s). Scientific names and common names will be required, and correct spelling will be emphasized. For each plant species on each quiz, plant genus will be assigned 4 points, species will be assigned 3 points, and the full common name (as taught) will be assigned 3 points. The plant identification quizzes will comprise 35\% of the student’s final grade, so students are strongly encouraged to focus on this aspect of the course on a daily basis.

\begin{fullwidth}

\textbf{Final Exam:} In addition to the plant identification quizzes administered at Cloquet and Itasca, one plant final exam will be administered at the Itasca Biological Station after moving from the Cloquet Forestry Center (on Saturday, August 24, 2019). The plant final exam will be ``closed book'' and no field notes or references will be allowed. Approximately two-thirds of the exam will cover plant identification (either on a short field hike or using collected live specimens in a lab setting), and the remaining one third will cover terminology, sampling techniques, plant community concepts, and other related material. The exam will be comprehensive, and may include all plant species and materials previously covered over the two weeks at the Cloquet Forestry Center. Scoring for species identification and correct spelling will be similar to that of the five plant identification quizzes.

\textbf{Community Narrative Paper:} Students will have one short written assignment of up to one page of written content. Students will be required to write a narrative description of one or more plant communities/habitats that they have observed. Plant community narratives will incorporate field data that students have collected during field sampling exercises, with reference to additional resources if relevant. More details on the written assignment will be provided in class, and past examples of narratives will be provided by instructors. Written assignments must be completed individually. The written assignment will be due on the morning of the third Friday, while at Itasca (unless otherwise specified). 

\textbf{Professionalism:} The August plants session is taught by two practicing professional ecologists and botanists. Students are expected to perform their work in the FW3106 in a highly professional manner. Students are expected to arrive on time, be prepared for field sessions, quizzes, and exams, collaborate with other students and instructors in group activities, and be respectful of others. Students are expected to record their observations in field notebooks for future reference and review. Field notebooks for FW3106 will not be graded. Professionalism of each student will be assessed by the instructors throughout the three week field session. 


Final grades will be determined in accordance with the University’s Uniform Grading Policy.

https://policy.umn.edu/education/gradingtranscripts
					

												
\newthought{Attendance:} Because of the compressed nature of this course, it is important that students attend all lectures and activities. Attendance is required. There is no good way to make up a field experience. The only acceptable reasons for missing any session during the class are: documented illness, documented family emergency, subpoenas, jury duty, military service, bereavement, and religious observances. Each unexcused absence will result in a 5\% deduction from your course total.

\newthought{Workload and Expectations:} This is a 1-credit course, which assumes 60 hours of work, according to University of Minnesota expectations. Students should therefore expect to spend 30 hours outside of class ($\sim$10 hours/week) in addition to the 30 hours of class time in order to master the material and complete assignments.

\newthought{Use of Personal Electronic Devices:} Please turn phones to off/mute during class sessions. Phones or cameras may be used to photograph plants and sampling sites while in field sessions, provided they do not become a distraction. When using phones and laptops, focus exclusively on the material at hand (i.e., no tweets, e-mail, or Facebook updates). You may not have your phone or smartwatch on you during the final exam.

\newthought{Student Conduct and Academic Integrity:} Academic integrity is essential to a positive teaching and learning environment. The University has strict policies regarding academic dishonesty, including cheating on exams and engaging in plagiarism. Students enrolled in University courses are expected to complete coursework responsibilities with fairness and honesty. Failure to do so by seeking unfair advantage over others or misrepresenting someone else’s work as your own, can result in disciplinary action. The University Student Conduct Code defines scholastic dishonesty as follows:

\emph{Scholastic dishonesty means plagiarizing; cheating on assignments or examination; engaging in unauthorized collaboration on academic work; taking, acquiring, or using test materials without faculty permission; submitting false or incomplete records of academic achievement; acting alone or in cooperation with another to falsify records or to obtain dishonestly grades, honors, awards, or professional endorsement; altering, forging, or misusing a University academic record; or fabricating or falsifying data, research procedures, or data analysis.}

A student responsible for scholastic dishonesty can be assigned a penalty up to and including an “F” or “N” for the course. If students have any questions regarding expectations for an assignment or exam, ask your course instructors.

For more information, see: http://www1.umn.edu/oscai/integrity/student/index.html

\newthought{Disability Accommodations:} Please see the course instructors during the first days of class if you require special accommodations because of a documented condition (physical, sensory, cognitive, systemic, learning, or psychiatric). The University has diagnostic services which may be helpful to you if you suspect that you have a condition that is hindering your academic performance. These services are available via Disabilities Services (612-626-1333, ds@umn.edu, http://www.mentalhealth.umn.edu/disability/index.html). You can also learn more about the broad range of confidential mental health services available on campus via the Student Mental Health Website: http://www.mentalhealth.umn.edu.

\newthought{University Policy on Sexual Harassment:} “Sexual harassment” means unwelcome sexual advances, requests for sexual favors, and/or other verbal or physical conduct of a sexual nature. Such conduct has the purpose or effect of unreasonably interfering with an individual’s work or academic performance or creating an intimidating, hostile, or offensive working or academic environment in any University activity or program. Such behavior is not acceptable in the University setting. 

For additional information, see:
https://safe-campus.umn.edu/sexual-misconduct-prevention

\newthought{Equity, Diversity, Equal Opportunity, and Affirmative Action:} The University will provide equal access to and opportunity in its programs and facilities, without regard to race color, creed, religion, national origin, gender, age, marital status, disability, public assistance status, veteran status, sexual orientation, gender identity, or gender expression.

\end{fullwidth}

\end{document}                              