\documentclass{tufte-handout}
\usepackage{fontspec}
\usepackage{termcal}

\makeatletter
\providecommand\tuftedate{}
\@ifpackageloaded{termcal}{%
  \renewcommand{\date}[1]{%
    \gdef\@date{#1}%
    \begingroup%
    % TODO store contents of \thanks command
    \renewcommand{\thanks}[1]{}% swallow \thanks contents
    \protected@xdef\tuftedate{#1}%
    \endgroup%
  }{%
    % Do nothing else, there's no need to redefine \date
  }
}
\makeatother
\defaultfontfeatures{Mapping=tex-text}

\renewcommand{\allcapsspacing}[1]{{\addfontfeature{LetterSpace=20.0}#1}}
\renewcommand{\smallcapsspacing}[1]{{\addfontfeature{LetterSpace=5.0}#1}}
\renewcommand{\textsc}[1]{\smallcapsspacing{\textsmallcaps{#1}}}
\renewcommand{\smallcaps}[1]{\smallcapsspacing{\scshape\MakeTextLowercase{#1}}}

\renewcommand{\calprintclass}{}


\title{Plant Identification \& Vegetation Sampling for Habitat Assessments}										% change each year
\author{FW 3106}										% change per section
\date{2019 August Field Session: 1 Credit}

\begin{document}



\maketitle


\newthought{Co-Instructor: Althea A.~ArchMiller\marginnote{The schedules and policies associated with this course may be subject to revision or change as a consequence of changing circumstances or events. Reasonable notification will be provided to students prior to any major changes in course policies or procedures.}}\\
218.556.8053 (cell)\\
Email: aarchmil@cord.edu\\
Twitter: @aaarchmiller\\


\newthought{Co-Instructor: Michael Lynch}\\
651.208.8734 (cell)\\
Email: mikassa.lynch@gmail.com\\

\begin{fullwidth}

\section{Course Description}

In this course, students will be introduced to common vegetation sampling methods used for habitat assessments. Students will learn to identify up to 110 plant species typical of Minnesota’s forest, prairie, wetland, and lacustrine ecosystems using field guides, taxonomic keys, and readily observable traits. Students will also learn about the importance of plant species composition, diversity, and structure in providing habitat for wildlife, as well as the role of ecological restoration and management in maintaining habitat quality. Coursework will include plant identification hikes, vegetation sampling exercises, participation and professionalism, one or more written assignments, plant identification quizzes, one vegetation sampling practical exam, and one plant identification final exam.

\subsection{Learning Outcomes}

Over the next three weeks, students will:

\begin{enumerate}
	\item Become familiar with plant traits commonly used to identify native plants, including leaf and stem morphology, growth form and structure, bark characteristics, and floral and fruit morphology.
	\item Learn to identify and recognize up to 110 common native and introduced plant species based on readily observable traits and habitat.
	\item Learn to use floristic keys and plant guides to correctly identify unknown plants and distinguish between similarly appearing species.
	\item Learn how to characterize vegetation using standard sampling methods, and compare the effectiveness of different methods.
	\item Understand basic native plant community concepts as they relate to fish and wildlife habitats.
	\item Understand the ways in which modern ecological land classifications integrate vegetation types to the larger geomorphic and climatic context where organisms inhabit.
\end{enumerate}

\subsection{Website}

Course materials can be accessed via the FW3108 Canvas website under the ``Plant Sampling Resources'' link at the bottom of the home page. Check the FW3108 Canvas site frequently to make sure you are current with FW3106 postings and resources.

\subsection{Communication}

The course instructors will use the regularly scheduled plant field sessions to communicate with the students regarding course announcements, schedules, and changes or modifications to the schedule or syllabus. Printed material will be handed out to the students at the beginning of these field sessions. Electronic versions of the handouts and additional supporting materials may be accessed on the FW3018 Canvas site under the Plant Sampling Resources link. In addition, the students may contact Althea ArchMiller and Michael Lynch via email or cell phone with additional course questions. Additional opportunities to learn the plant material and other course material will be offered outside of regularly scheduled field sessions in the form of plant walks or study sessions. Students are encouraged to request assistance from Althea ArchMiller and Michael Lynch throughout the August field session.

\subsection{Texts} 

The required textbook for this course is \emph{Key Plants Appearing in the Field Guides to Native Plant Communities of Minnesota: Forests \& Woodlands}, Second Edition (Almendinger 2015), and is available for download as a PDF from the course Canvas site. All students must download this plant identification manual and use it as their reference for reviewing plants taught in the field during the plant identification hikes. There are also four printed copies of this manual available for student use in the plant resource library in the plant lab. Supplemental resources and readings will be made available via the course website and resource library. Plant lists and supporting material will be distributed to all students during the scheduled field sessions.

\newthought{Students are strongly encouraged} to bring at least one additional plant field guide of their choice. Students are encouraged to purchase and reference the publication \emph{The Field Guide to the Native Plant Communities of Minnesota – The Laurentian Mixed Forest Province} (MN DNR, 2003, available at the Minnesota Bookstore: https://mn.gov/admin/bookstore), which now serves as the preferred field reference for native vegetation and plant community classification of the Cloquet and Itasca landscapes. 

Finally, the Minnesota Wildflowers app is free and available for most smart phones. This is a nice way to identify unknown flowering plants and grasses in Minnesota. 

\subsection{Attendance}

Because of the compressed nature of this course, it is important that students attend all lectures and activities. Attendance is required. There is no good way to make up a field experience. The only acceptable reasons for missing any session during the class are: documented illness, documented family emergency, subpoenas, jury duty, military service, bereavement, and religious observances. Each unexcused absence will result in a 5\% deduction from your course total.

This is a 1-credit course, which assumes 60 hours of work, according to University of Minnesota expectations. Students should therefore expect to spend 30 hours outside of class ($\sim$10 hours/week) in addition to the 30 hours of class time in order to master the material and complete assignments.

\subsection{Respect for Diversity}

It is our intent that students from diverse backgrounds and perspectives be well-served by this course, and that the diversity that students bring to this class be viewed as a resource. Please let us know ways to improve the effectiveness of the course for you, personally, or for other students or student groups. As a student in this class, you are required to treat other members of the class with respect and kindness. Disrespectful, rude, or exclusive behavior will not be tolerated.

\section{Grades}

Class sessions will typically include a plant identification hike and/or vegetation sampling exercises. The course instructors will introduce key concepts and terminology via infrequent brief lectures, and occasionally we will re-enforce certain skills via classroom activities, but the majority of our class time will be outdoors. All students will be evaluated based on the following components.

%Final grades will be based on the following:

\begin{table}
\begin{tabular}{l l  r r}
Item & Details & Points & \% \\
\hline
Plant Sampling Activities &  & 100 & 10 \\
Plant Sampling Practical Exam & Aug.\ 19 or 20 & 100 & 10 \\
Plant Identification Quizzes & Up to 7 quizzes & 350 & 35 \\
Final Exam & Aug.\ 24 & 250  & 25 \\						% change for each unit
Community Narrative Paper & Due Aug.\ 30 & 100 & 10 \\
Professionalism & & 100 & 10 \\
%& Homework & varying dates & 5 \\
\hline
& & Total 1000 & 100
\end{tabular}
\end{table}

\end{fullwidth}

\textbf{Plant Sampling Exercises:} Students will work in pairs or groups of 6 to complete 2 to 3 sampling activities in class during the three week field session, including an aquatic plant survey at Lake Ozawindib. Students will turn in completed data sheets at the end of each exercise (unless otherwise specified). Vegetation sampling is an important comprehensive skill to understand while completing the FW3106 coursework. 

\textbf{Plant Identification Quizzes} will be given at the beginning of most class sessions, other than the first, during which we will ask students to identify 10 species taught in the previous field session(s). 



are from the interactive textbook for this class, and each module has integrated, feedback-focused questions followed by a series of graded questions. \textbf{You are expected to have read that day's SimUText material prior to coming to class. } SimUText graded questions are due by 8:00am on the due date (see schedule).

\textbf{Reading Completion} will be evaluated with the feedback-focused, ungraded questions and will be assessed with a pass-fail grade (completed or not) for each SimUText assignment. 

\begin{margintable}
\begin{tabular}{rl}
Percentage & Grade \\
\hline 
$\ge93$ & A \\
90-92.9 & A- \\
87-89.9 & B+ \\
83-86.9 & B \\
80-82.9 & B- \\
77-79.9 & C+ \\
73-76.9 & C \\
70-72.9 & C- \\
67-69.9 & D+ \\
60-66.9 & D \\
$<60$ & F \\
\hline
\end{tabular}
\end{margintable}


\textbf{Graded Questions} will be worth another 5\% of your final grade; however, the two lowest scores will be dropped before final grades are completed. You may work through the SimUText material with your peers; however, mastering the material is your individual responsibility.



\begin{fullwidth}	

\newthought{{Quizzes}} are designed to quickly check for reading and comprehension of that lecture date's SimUText material. Quizzes will be short ($\sim$3 questions) and given at the beginning of class time on most days. I will drop the two lowest quiz scores.

$^*$In addition, I will make homework available for students that have excused absenses. If you have an excused absense (thus a 0 for that quiz), you may---up to 3 times over the course of the semester---complete homework to replace a zero quiz score. The homework assignments will be designed to give you more hands-on practice with quantitative topics covered in lecture and in the SimUText readings; however, they will be more difficult than quizzes.
	
\newpage 

\newthought{{Lecture Exams}} will be of variable format, including---but not limited to---multiple choice, true/false, matching, short answer, and brief essays. All exams will be somewhat cumulative but will primarily focus on the associated SimUText Unit material (see table above); in addition, the final exam will be one-third cumulative. 
					
												
\newthought{{Group Discussions}} allow you to work as a team of scientists with your colleagues to critically discuss three separate books, \emph{The Serengeti Rules}, \emph{The Omnivore's Dilemma}, and \emph{A Sand County Almanac}. Group discussions will occur in forum format on Moodle. Groups will be assigned at random and will be reassigned for each new book 
(i.e., by October 19 and November 9). 											% change for each year
You will be graded based on the quantity, quality and timing of your comments (see grading rubric below). Each discussion is worth a total of 5 points.

The group as a whole is responsible for completing the assignment; in this case providing a good discussion and coming to a better understanding of ecology and evolution. Everyone should contribute to the discussion, and you are expected to provide at least two comments; ideally one will be an original question or discussion point, and one will be a reply to another group member's comment. You should take this opportunity to learn from and respectfully teach each other.

%\begin{table}
\begin{tabular}{l l l l}
\\
\hline
\textbf{Grading Criteria} & \textbf{Exemplary} & \textbf{Adequate} & \textbf{Poor} \\
\hline
Quantity of Comments & $>$2 Comments & 2 Comments & 1 Comment \\
& (2pts) & (1.5pts) & (1pt) \\
\hline
Quality of Comments & Focused on ecological & Indicated a superficial & Conveyed little \\
& aspects and tackled & understanding of reading & understanding of reading; \\
& central themes of & or focused on details w/o& not relevant to ecology \\
& reading &  conveying importance to & or main themes of text \\
& & ecology or main themes & \\
& & of text & \\
& (2pts) & (1.5pts) & (1pt) \\
\hline
Timing of 1$^\mathrm{st}$ Comment & $>$48 hrs before due & 24--48 hrs before due & $<$24 hrs before due \\
& (1pt) & (0.5pt) & (0pts) \\
\hline \\
\end{tabular}
%\end{table}

\newthought{{The Symposium Paper}} is a 3-page, 1.5-spaced, 12-pt font paper, that is due at 
11:55pm on Wednesday, September 26 (upload on Moodle). 											% change each year
The 2018 Symposium, Power Plays: Why Gender Matters, % change each year
takes place on September 18--19, 													 % change each year
and you are required to attend. The Symposium Paper should name and summarize the session you attended, including questions/answers raised during the Q/A of the session, and your reaction. At least one page of your paper should explore how the symposia relate to ecology, the environment and campus life. You will be graded out of 100 points based on the following (detailed rubric is on Moodle): 

\begin{itemize}
\item Spelling and grammar (20pts)
\item Summary of session and Q/As (40pts)
%\item Relation of session topic to ecology and the environment (30pts)
\item Relation of session topic to campus life and science (40pts)
\end{itemize}

%\newthought{\textbf{Homework}} will be designed to give you more hands-on practice with quantitative topics covered in lecture and in the SimUText readings. The due date and timewill be specified on each homework assignment. 

\newpage


\end{fullwidth}

\section{Academic Integrity (from Student Handbook)}

\marginnote{Concordia College has university-wide policies about academic integrity, and all students are responsible for being familiar with and adhering to them. These policies are in place to protect students, first and foremost. \textbf{My role as instructor is to teach each of my students how to become responsible scholars.} As a student at Concordia College and as a student in this class, you are expected to fully and properly acknowledge the work of others. Every instance of plagiarism will be reported, as per the policies of the college, but please do not hesitate to ask me in advance if you think something might be questionable or if you are unsure about what is considered to be plagiarism. I am happy to help, as long as you inquire in advance! }

``The Concordia community expects all of our members to act with integrity--to act with honesty, uprightness and sincerity. Every member of our academic community is charged with the responsibility of encouraging and maintaining an environment of academic integrity.

``Academic misconduct is defined as any activity that comprises the academic integrity of the college or undermines the educational process. Academic misconduct includes but is not limited to:

\begin{itemize}
	\item cheating: using a resource other than one's own work to answer questions;
	\item plagiarism: misrepresenting another's ideas as one's own or not giving credit to the creator of a work;
	\item falsification: submitting falsified or fabricated information;
	\item facilitating others' violations: knowingly permitting or facilitating the dishonesty of others;
	\item impeding: placing barriers in the way of others' academic pursuits''
\end{itemize}

\begin{fullwidth}

\subsection{Biology Department policy on use of electronic devices (phones, smart watches, laptops, tablets, etc.)}

Faculty in the Biology Department work to make the classroom and laboratory a space conducive to student learning. We encourage writing notes by hand because it is an effective learning strategy for many students. However, the Biology Department also understands the valuable role of electronic devices in learning and scholarship. Thus, the Biology Department policy on the use of these devices in the classroom is as follows:


\begin{enumerate}
\item Electronic devices used during class time should be limited to appropriate class-related activities as outlined by the instructor. We reserve the right to check devices at any time and to ask you to put them away or leave if we see you using them inappropriately. Please reduce distractions to yourself and your fellow classmates.
\item All electronic devices must be set to silent during scheduled classroom and laboratory sessions. Tones and vibrations are distracting.
\item Only approved electronic devices (such as non-programmable calculators) may be available or used during examination periods. We expect that all non-approved electronic devices will be turned off and stored away from the exam areas.
\item Sharing calculators during exams is not allowed without permission. 
\item Cheating in any form, including through use of an electronic device, will not be tolerated. See the academic integrity policy for more information.
\end{enumerate}

Inappropriate or distracting use of electronic devices in the classroom may adversely affect your course grade. 

\subsection{Accommodations for Students with Disabilities}

In accordance with the Americans with Disabilities Act, Concordia College and your instructor are committed to making reasonable accommodations to assist individuals with documented disabilities to reach their academic potential. Such disabilities include, but are not limited to, learning or psychological disabilities, or impairments to health, hearing, sight, or mobility. If you believe you require accommodations for a disability that may impact your performance in this course, you must schedule an appointment with Disability Services to determine eligibility. Students are then responsible for giving instructors a letter from Disability Services indicating the type of accommodation to be provided; please note that accommodations will not be retroactive. The Disability Services office is in Academy 106, phone 218-299-3514; https://www.concordiacollege.edu/directories/offices-services/counseling-center-and-disabilityservices/disability/ 



\newpage

\section{Course Schedule (version dated 8/29/2018)}

\begin{itemize}
	\item SimUText Sections: You are expected to come to class prepared by reading that lecture's associated SimUText Module.  There will be quizzes on reading material at the beginning of lecture.
	\item GD: Group discussions on Moodle. You will be graded based on your participation and are expected to post to each discussion section at least twice by 5pm the day each discussion unit is due.
	%\item OH: Dr.~ArchMiller's office hours, ISC 224 (MWF, but at varying times; see schedule for details)
\end{itemize}



  \setlength{\calwidth}{6.5in}
  \setlength{\calboxdepth}{0.3in}
  \begin{calendar}{9/3/18}{16}

  \calday[Monday]{\classday} % Monday
  \calday[Tuesday]{\classday} % Wednesday
  \calday[Wednesday]{\classday}
  \calday[Thursday]{\classday} % Thursday (unnumbered)
  \calday[Friday]{\classday} % Friday
    \skipday\skipday % weekend (no class)


% Week 1

\caltext{9/3/18}{{SimUText Unit 1:} Evolution for Ecology 1-3, Biogeography 3}
\caltext{9/7/18}{\textbf{GD:} The Serengeti Rules p1--46 (by 5pm) }

% Week 2
\caltext{9/10/18}{Library Materials Lecture \emph{(upload assignment on Lab Moodle page)}}
\caltext{9/12/18}{{SimUText Unit 1:} Biogeography 4, Physiological Ecology 1}
\caltext{9/14/18}{\textbf{GD:} The Serengeti Rules p47--105 (by 5pm) }
\caltext{9/14/18}{\textbf{Library Materials assignment due} on Lab Moodle page by 11:55pm}

% Week 3
\caltext{9/17/18}{{SimUText Unit 1:} Physiological Ecology 2-4}
\caltext{9/19/18}{\textbf{Symposium} \\ No office hours}

% Week 4
\caltext{9/24/18}{{SimUText Unit 1:} Ecosystem Ecology 1-3}
\caltext{9/26/18}{\textbf{Symposium Paper due} on Moodle by 11:55pm}
\caltext{9/28/18}{\textbf{GD:} The Serengeti Rules p107--168 (by 5pm) }

% Week 5
\caltext{10/1/18}{{SimUText Unit 1:} Climate Change 1-5}
\caltext{10/5/18}{\textbf{GD:} The Serengeti Rules p169--214 (by 5pm) }

% Week 6
\caltext{10/8/18}{\textbf{Unit 1 SimUText Graded Questions Due at 8am}}
\caltext{10/10/18}{\textbf{EXAM 1}}

% Week 7
\caltext{10/15/18}{SimUText Unit 2: Nutrient Cycling 1-4}
\caltext{10/19/18}{\textbf{GD:}  Omnivore's Dilemma p1-56 (by 5pm) }

% Week 8
\caltext{10/22/18}{\emph{Mid Semester Break--No Class}}
\caltext{10/23/18}{\emph{Mid Semester Break--No Class}}
\caltext{10/24/18}{\emph{Mid Semester Break--No Class}}
\caltext{10/25/18}{\emph{Mid Semester Break--No Class}}
\caltext{10/26/18}{\emph{Mid Semester Break--No Class}}

% Week 9
\caltext{10/29/18}{SimUText Unit 2: Life History 1-4}
\caltext{11/2/18}{\textbf{GD:}  Omnivore's Dilemma p65-99, p410-411 (5pm) }

% Week 10

\caltext{11/5/18}{SimUText Unit 2: Population Growth 1-3}
\caltext{11/5/18}{\textbf{In Class:} Understanding Population Growth Models}
\caltext{11/7/18}{SimUText Unit 2: Population Growth 4-5}
\caltext{11/9/18}{\textbf{GD:}  Sand County Almanac pp vii--52 (by 5pm) }

% Week 11
\caltext{11/12/18}{\textbf{Unit 2 SimUText Graded Questions Due at 8am}}
\caltext{11/14/18}{\textbf{EXAM 2}}


% Week 12
\caltext{11/19/18}{SimUText Unit 3: Biogeography 1-2}
\caltext{11/20/18}{\textbf{GD:}  Sand County Almanac p53--92 (by 5pm) }
\caltext{11/21/18}{\emph{Thanksgiving--No Class}}
\caltext{11/22/18}{\emph{Thanksgiving--No Class}}
\caltext{11/23/18}{\emph{Thanksgiving--No Class}}

% Week 13
\caltext{11/26/18}{SimUText Unit 3: Community Dynamics 1-2}
\caltext{11/28/18}{SimUText Unit 3: Community Dynamics 3-5}
\caltext{11/30/18}{\textbf{GD:}  Sand County Almanac p95-112; 129--137; 165--176 (by 5pm) }

% Week 14
\caltext{12/3/18}{SimUText Unit 3: Competition 1-2}
\caltext{12/5/18}{SimUText Unit 3: Competition 3-4}
\caltext{12/7/18}{\textbf{GD:} Sand County Almanac p188--226 (by 5pm) }

% Week 15
\caltext{12/10/18}{SimUText Unit 3: Predation, Herbivory and Parasitism 1-2}
\caltext{12/12/18}{SimUText Unit 3: Predation, Herbivory and Parasitism 3-4}
\caltext{12/14/18}{Special review 8:00-9:20am}
\caltext{12/14/18}{\textbf{Unit 3 SimUText Graded Questions Due at 8am}}

% Week 16
\caltext{12/17/18}{\textbf{FINAL EXAM 8:30-10:30am}}							% change by section


  \end{calendar}



\end{fullwidth}



%\newpage

%Intentionally left blank

\newpage

\subsection{Syllabus Acknowledgement}

I, \underline{\hspace{5cm}}, have received a copy of the syllabus for BIOL 221, Ecology, and understand all of the policies and procedures outlined herein. 

\newthought{Signature}  \underline{\hspace{5cm}} {Date}  \hrulefill


\subsection{Use of Photographic Likeness Release}

For good and valuable consideration, I authorize Dr.~ArchMiller to record photographs of me and use, reproduce, modify, distribute, and exhibit such photographs, in whole or in part, without restrictions or limitation for marketing and instructional purposes. 

I release Dr.~ArchMiller, Concordia College, its successors and assigns, agents, and all persons for whom it is acting from any liability by virtue of any blurring, distortion, alteration, optical illusion, or use in composite form, whether intentional or otherwise, that may occur or be produced in the photographic process and waive any right that I may have to inspect or approve the finished recordings.

\newthought{Printed name}  \hrulefill
\newthought{Signature}  \underline{\hspace{5cm}} {Date}  \hrulefill

\subsection{Optional Information}

\newthought{Preferred name or nickname} \hrulefill

\newthought{Major} \hrulefill

\newthought{Contact phone number} \hrulefill

\newthought{Where's ``home?''} \hrulefill


\newthought{Ways you are similar to other Cobbers} \hrulefill

\hrulefill

\hrulefill

\newthought{Ways you are unique compared to other Cobbers}\hrulefill

\hrulefill

\hrulefill
\end{document}                              