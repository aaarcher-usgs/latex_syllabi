\documentclass{tufte-handout}
\usepackage{fontspec}
\usepackage{termcal}

\makeatletter
\providecommand\tuftedate{}
\@ifpackageloaded{termcal}{%
  \renewcommand{\date}[1]{%
    \gdef\@date{#1}%
    \begingroup%
    % TODO store contents of \thanks command
    \renewcommand{\thanks}[1]{}% swallow \thanks contents
    \protected@xdef\tuftedate{#1}%
    \endgroup%
  }{%
    % Do nothing else, there's no need to redefine \date
  }
}
\makeatother
\defaultfontfeatures{Mapping=tex-text}

\renewcommand{\allcapsspacing}[1]{{\addfontfeature{LetterSpace=20.0}#1}}
\renewcommand{\smallcapsspacing}[1]{{\addfontfeature{LetterSpace=5.0}#1}}
\renewcommand{\textsc}[1]{\smallcapsspacing{\textsmallcaps{#1}}}
\renewcommand{\smallcaps}[1]{\smallcapsspacing{\scshape\MakeTextLowercase{#1}}}

\renewcommand{\calprintclass}{}

\title{2020 Syllabus for Biology 312: General Ecology}										% change each year
\author{Lecture: Monday/Wednesday/Friday 10:00am--10:50am}										% change per section
\date{Lab: Thursday 8-10:50}

\begin{document}
\maketitle

Instructor: Dr.~Althea A.~Archer\marginnote{The schedules and policies associated with this course may be subject to revision or change as a consequence of changing circumstances or events. Reasonable notification will be provided to students prior to any major changes in course policies or procedures.}\\
Office: 267 Wick Science Building\\
320-308-4975 (office) / 218.556.8053 (cell)\\
Email: althea.archer@stcloudstate.edu\\
Twitter: @aaarchmiller

Virtual Office Hours: Mon/Fri 12:15-1:15pm \& Th 11:30-12:30\\
Link: https://minnstate.zoom.us/j/99287589339\\
Meeting ID: 992 8758 9339\\
Passcode: Archer


\begin{fullwidth}

\section{Course Description}

Interactions between organisms and their organic and inorganic environment. Biomes, climate, populations, communities, biotic interactions, energy and nutrients, landscape and spatial ecology, biodiversity patterns.

\subsection{Learning Outcomes}

You will learn to draw together elements from biology, chemistry, physics, geology, and mathematics to gain a greater understanding of ecological relationships in the natural world. The goals of the course are to:


\begin{enumerate}
	%\item Access, critically evaluate, and correctly use scientific literature
	\item Classify organizational levels observed in ecology
	\item Explain how populations are regulated and how data can be collected, analyzed, and interpreted using statistics, life tables, graphs, and survivorship curves
	\item Describe the interactions between different species and how they impact one another
	\item Illustrate the major forces responsible for community structure, how community structure can be represented by food webs, and how communities change in both space and time
	\item Discuss patterns and measurements of biodiversity and predict the consequences of continued species loss
	\item Accurately and effectively document field observations with field notes and data collection
	\item Link field observations with key ecological concepts and relevant scientific literature
	\item Execute the scientific method using reproducible research methods
	\item Effectively communicate scientific research results through oral and written presentations
\end{enumerate}

\subsection{Required Textbooks}

\begin{itemize}
	\item SimUText Ecology
	\item Recommended: McMillan, V.E. 2012. \emph{Writing Papers in the Biological Sciences}. 5th ed. New York: Bedford/St. Martin's
	\item Recommended: Molles, Jr., M.C. \emph{Ecology: Concepts and Applications}.
\end{itemize}

\subsection{Attendance Policy}

Regular attendance and participation in class is critical to your success. This course will be offered in a hybrid format. Most of the lectures will be convened online via synchronous Zoom meetings, and the textbook assignments will be conducted through an interactive online textbook. The first five labs will require in-person activities in an outdoor setting. You will be working with small groups during each lab, and will be required to wear a mask. 

\textbf{Students/faculty/staff displaying signs and symptoms compatible with COVID-19 (fever/chills, cough, shortness of breath, fatigue, body aches, headache, loss of smell/taste, sore throat, congestion,  nausea/vomiting, diarrhea) will be asked to stay home (not enter the classroom) and will be encouraged to self-isolate/seek medical attention.}



\subsection{Accommodations for Students with Disabilities}

If you have a mental or physical disability and qualify for accommodations under the Americans With Disabilities Act (for instance, additional time for exams), please contact the SCSU Student Disability Services office. If notified that a student qualifies for accommodations, I will make individual arrangements.

\subsection{Respect for Diversity}

It is my intent that students from diverse backgrounds and perspectives be well-served by this course, and that the diversity that students bring to this class be viewed as a resource. Please let me know ways to improve the effectiveness of the course for you, personally, or for other students or student groups. As a student in this class, you are required to treat other members of the class with respect and kindness. Disrespectful, rude, or exclusive behavior will not be tolerated.

\end{fullwidth}

\newthought{Grades}

%Final grades will be based on the following:
\begin{fullwidth}

\begin{table}
\begin{tabular}{l l l r}
Category & Item & Details & \% \\
\hline
SimUText Readings &  & various dates & 10\\
%Participation & & throughout semester & 5\\
\hline
Lecture Exams \\
& Exam 1 & Oct.~9; Unit 1 material & 15 \\
& Exam 2 & Nov.~13; Unit 2 material & 15 \\
& Final Exam & Dec.~17; 66\% Unit 3; 34\% Units 1\&2 & 20 \\ 							% change for each unit
\hline
Laboratory & &  &  \\
& Data Sheets & Sept.~24 & 5 \\
& Data Appendix & Oct.~8 & 5 \\
& Lightning Talk & Nov.~5 & 10 \\
%& Research Abstract & Nov.~19 & 5 \\
& Research Poster Draft & Dec.~3 & 5 \\
& Research Poster Final & Dec.~10 & 10 \\
& Peer Feedback & various dates & 5 \\
\hline
& & & Total 100
\end{tabular}
\end{table}

\end{fullwidth}

\newpage

\newthought{SimUText Readings} are from the interactive textbook for this class, and each module has integrated, feedback-focused questions followed by a series of graded questions. \textbf{You are expected to have read that day's SimUText material prior to coming to class. } SimUText graded questions are due by 10:00pm on the due date, which is usually Friday (see schedule). You may work through the SimUText material with your peers; however, mastering the material is your individual responsibility.


\begin{margintable}
\begin{tabular}{rl}
Percentage & Grade \\
\hline 
$\ge93$ & A \\
90-92.9 & A- \\
87-89.9 & B+ \\
83-86.9 & B \\
80-82.9 & B- \\
77-79.9 & C+ \\
73-76.9 & C \\
70-72.9 & C- \\
67-69.9 & D+ \\
60-66.9 & D \\
$<60$ & F \\
\hline
\end{tabular}
\end{margintable}

\newthought{{Lecture Exams}} will be of variable format, including---but not limited to---multiple choice, true/false, matching, short answer, and brief essays. All exams will be somewhat cumulative but will primarily focus on the associated lecture and SimUText Unit material (see table above); in addition, the final exam will be one-third cumulative. 

%\textbf{Graded Questions} will be worth another 5\% of your final grade; however, the two lowest scores will be dropped before final grades are completed. 
\begin{fullwidth}

\newthought{Laboratory} grades will be based around a semester-long group research project that will begin with collecting data in the field, continue with data entry, organization, and analysis, and culminate in oral and written presentations. 

\begin{itemize}
\item Data sheets will be completed in the field during each of the first 5 labs. All data sheets will be due by end of the fifth lab, although I recommend that you hand in your datasheets at the end of each lab. I also highly recommend scanning in datasheets with your phone and emailing them to all group mates at the end of each lab. 
\item Data appendix will be an html document that includes summary statistics about each of the variables relevant to your research project and dataset. A template and further explanation will be provided later in the semester. 
\item Lightning talks will be given during lab. Your group will be allowed 3 slides and 5 minutes to present the main goal, result, and conclusion of your research project. You will be providing feedback to other groups, which will go toward your ``Peer Feedback" grade, and you will be expected to incorporate feedback into your final poster presentation.
\item Research poster must include title, introduction, methods, results, discussion, conclusions, and literature cited. A rubric for posters will be provided later this semester. Every group will present their poster draft in the penultimate lab session. You will be providing feedback to other groups on their posters, which will go towards your ``Peer Feedback'' grade, and you will be expected to incorporate feedback into your final poster. 
\item The final (virtual) research poster presentation will be open to friends and family outside of the class.
\item Your peer assessment grade will include the quality of your formal feedback during the lightning talks and the draft poster presentation (33\% each) and the grade that your group mates give you at the culmination of the semester (34\%). 
\end{itemize}

\newthought{Participation} in class and lab will not go towards your grade directly. However, a record throughout the semester of exemplary participation and attendance can help in the case of a borderline final grade. Active participation also nurtures learning, and will improve the quality of future recommendation letters from your instructors.  

\end{fullwidth}


\section{Academic Integrity}

%\marginnote{Concordia College has university-wide policies about academic integrity, and all students are responsible for being familiar with and adhering to them. These policies are in place to protect students, first and foremost. \textbf{My role as instructor is to teach each of my students how to become responsible scholars.} }

%``The Concordia community expects all of our members to act with integrity--to act with honesty, uprightness and sincerity. Every member of our academic community is charged with the responsibility of encouraging and maintaining an environment of academic integrity.

%``Academic misconduct is defined as any activity that comprises the academic integrity of the college or undermines the educational process. Academic misconduct includes but is not limited to:

\marginnote{As a student at St.\ Cloud State University and as a student in this class, you are expected to fully and properly acknowledge the work of others. Every instance of plagiarism will be reported, as per the policies of the college, but please do not hesitate to ask me in advance if you think something might be questionable or if you are unsure about what is considered to be plagiarism. I am happy to help, as long as you inquire in advance! }

Academic misconduct includes but is not limited to:

\begin{itemize}
	\item cheating: using a resource other than one's own work to answer questions;
	\item plagiarism: misrepresenting another's ideas as one's own or not giving credit to the creator of a work;
	\item falsification: submitting falsified or fabricated information;
	\item facilitating others' violations: knowingly permitting or facilitating the dishonesty of others;
	\item impeding: placing barriers in the way of others' academic pursuits''
\end{itemize}

\begin{fullwidth}






%\newpage

\section{Course Schedule (version dated 8/14/2020)}



  \setlength{\calwidth}{6.5in}
  \setlength{\calboxdepth}{0.3in}
  \begin{calendar}{8/24/20}{16}

  \calday[Monday]{\classday} % Monday
  \calday[Tuesday]{\classday} % Wednesday
  \calday[Wednesday]{\classday}
  \calday[Thursday]{\classday} % Thursday (unnumbered)
  \calday[Friday]{\classday} % Friday
    \skipday\skipday % weekend (no class)


% Week 1
\caltext{8/24/20}{First day of class}
\caltext{8/26/20}{\textbf{Topic:} Introduction to Ecology}
\caltext{8/27/20}{\textbf{Lab:} Guided tour}
\caltext{8/28/20}{\textbf{Topic:} Experimental Design}
\caltext{8/28/20}{\emph{SimUText Unit 1: Understanding Experimental Design Due 10pm}}


% Week 2
\caltext{8/31/20}{\textbf{Topic:} Evolution for Ecology 1}
\caltext{9/2/20}{\textbf{Topic:} Evolution for Ecology 2}
\caltext{9/3/20}{\textbf{Lab:} Macroinvertebrate Survey}
\caltext{9/4/20}{\textbf{Topic:} Evolution for Ecology 3}
\caltext{9/4/20}{\emph{SimUText Unit 1: Evolution for Ecology 1-3 due 10pm}}




% Week 3
\caltext{9/7/20}{\emph{No class}}
\caltext{9/9/20}{\textbf{Topic:} Biogeography 3}
\caltext{9/10/20}{\textbf{Lab:} Pollinator Survey}
\caltext{9/11/20}{\textbf{Topic:} t-test, ANOVA, regression}
\caltext{9/11/20}{\emph{SimUText Unit 1: Biogeography 3 due 10pm}}


% Week 4
\caltext{9/14/20}{\textbf{Topic:} Behavioral Ecology 1-2}
\caltext{9/16/20}{\textbf{Topic:} Behavioral Ecology 3-4}
\caltext{9/17/20}{\textbf{Lab:} Prairie Diversity Survey}
\caltext{9/18/20}{\textbf{Topic:} Behavioral Ecology 5}
\caltext{9/18/20}{\emph{SimUText Unit 1: Behavioral Ecology 1-5 due 10pm}}


% Week 5
\caltext{9/21/20}{\textbf{Topic:} Biogeography 4, Physiological Ecology 1}
\caltext{9/23/20}{\textbf{Topic:} Biogeography 4, Physiological Ecology 1 (cont.)}
\caltext{9/24/20}{\textbf{Lab:} Forest Survey}
\caltext{9/25/20}{\textbf{Topic:} Physiological Ecology 2}
\caltext{9/25/20}{\emph{SimUText Unit 1: Biogeography 4, Physiological Ecology 1-2 due 10pm}}

% Week 6
\caltext{9/28/20}{\textbf{Topic:} Physiological Ecology 3}
\caltext{9/30/20}{\textbf{Exam 1}}
\caltext{10/1/20}{\textbf{Lab:} Introduction to R \& How Diseases Spread \emph{SimUText Lab: How Diseases Spread due 10pm}}
\caltext{10/2/20}{\textbf{Topic:} Physiological Ecology 4}
\caltext{10/2/20}{\emph{SimUText Unit 1: Physiological Ecology 3-4 due 10pm}}

% Week 7
\caltext{10/5/20}{\textbf{Topic:} Ecosystem Ecology 1-2}
\caltext{10/7/20}{\textbf{Topic:} Ecosystem Ecology 3}
\caltext{10/8/20}{\textbf{Lab:} Data Appendix due 10pm (OSF)}
\caltext{10/9/20}{\emph{No class}}


% Week 8
\caltext{10/12/20}{\textbf{Topic:} Ecosystem Ecology 1-2}
\caltext{10/14/20}{\textbf{Topic:} Ecosystem Ecology 3}
\caltext{10/15/20}{\textbf{Lab:} Data analysis}
\caltext{10/16/20}{\textbf{Topic:} Ecosystem Ecology 4}
\caltext{10/16/20}{\emph{SimUText Unit 1: Ecosystem Ecology 1-4 due 10pm}}

% Week 9
\caltext{10/12/20}{\textbf{Topic:} Ecosystem Ecology 1-2}
\caltext{10/14/20}{\textbf{Topic:} Ecosystem Ecology 3}
\caltext{10/15/20}{\textbf{Lab:} Data analysis}
\caltext{10/16/20}{\textbf{Topic:} Ecosystem Ecology 4}
\caltext{10/16/20}{\emph{SimUText Unit 1: Ecosystem Ecology 1-4 due 10pm}}


% Week 10

\caltext{11/4/19}{SimUText Unit 2: Population Growth 1-3}
\caltext{11/4/19}{\textbf{In Class:} Understanding Population Growth Models}
\caltext{11/6/19}{SimUText Unit 2: Population Growth 4-5}
\caltext{11/6/19}{\textbf{Discussion:} Serengeti  ch 3-5}
\caltext{11/8/19}{\textbf{Response:} Serengeti Rules ch 3-5}

% Week 11
\caltext{11/11/19}{SimUText Unit 2: Biogeography 1-2}
\caltext{11/11/19}{\textbf{Unit 2 SimUText Graded Questions Due at 8am}}
\caltext{11/13/19}{\textbf{EXAM 2}}


% Week 12
\caltext{11/18/19}{SimUText Unit 3: Community Dynamics 1-2}
\caltext{11/20/19}{SimUText Unit 3: Community Dynamics 3-5}
\caltext{11/20/19}{\textbf{Discussion:} Serengeti  ch 6-8}
\caltext{11/22/19}{\textbf{Response:} Serengeti Rules ch 6-8}

% Week 13
\caltext{11/25/19}{SimUText Unit 3: Competition 1-2}
\caltext{11/27/19}{\emph{Thanksgiving}}
\caltext{11/28/19}{\emph{Thanksgiving}}
\caltext{11/29/19}{\emph{Thanksgiving}}

% Week 14
\caltext{12/2/19}{SimUText Unit 3: Competition 3-4}
\caltext{12/4/19}{SimUText Unit 3: Predation, Herbivory and Parasitism 1-2}
\caltext{12/4/19}{\textbf{Discussion:} Serengeti  ch 9-10}
\caltext{12/6/19}{\textbf{Response:} Serengeti Rules ch 9-10}

% Week 15

\caltext{12/9/19}{SimUText Unit 3: Predation, Herbivory and Parasitism 3-4}
\caltext{12/11/19}{\textbf{Unit 3 SimUText Graded Qs Due at 8am}}

% Week 16
\caltext{12/17/19}{\textbf{FINAL EXAM 8:30-10:30am}}							% change by section



  \end{calendar}



\end{fullwidth}



%\newpage

\end{document}                              