\documentclass{tufte-handout}
\usepackage{fontspec}
\usepackage{termcal}

\makeatletter
\providecommand\tuftedate{}
\@ifpackageloaded{termcal}{%
  \renewcommand{\date}[1]{%
    \gdef\@date{#1}%
    \begingroup%
    % TODO store contents of \thanks command
    \renewcommand{\thanks}[1]{}% swallow \thanks contents
    \protected@xdef\tuftedate{#1}%
    \endgroup%
  }{%
    % Do nothing else, there's no need to redefine \date
  }
}
\makeatother
\defaultfontfeatures{Mapping=tex-text}

\renewcommand{\allcapsspacing}[1]{{\addfontfeature{LetterSpace=20.0}#1}}
\renewcommand{\smallcapsspacing}[1]{{\addfontfeature{LetterSpace=5.0}#1}}
\renewcommand{\textsc}[1]{\smallcapsspacing{\textsmallcaps{#1}}}
\renewcommand{\smallcaps}[1]{\smallcapsspacing{\scshape\MakeTextLowercase{#1}}}

\renewcommand{\calprintclass}{}

\title{BIOL 122: Evolution \& Diversity 2020 -- Syllabus}										% change each year
\author{20017: Mon/Wed 7:50-10:15am \& Fri 8:00-9:10am - ISC256}										% change per section
\date{}

\begin{document}
\maketitle

Instructor: Dr.~Althea A.~ArchMiller\marginnote{The schedules and policies associated with this course may be subject to revision or change as a consequence of changing circumstances or events. Reasonable notification will be provided to students prior to any major changes in course policies or procedures.}\\
Office: 220 Integrated Science Center\\
Phone: 218.299.3793\\
Email: aarchmil@cord.edu\\
Twitter: @aaarchmiller\\
Office Hours:  MW 10:30am-12:00pm; T 8-9:30am\\


\begin{fullwidth}

\subsection{Course Goals}

Students in Evolution and Diversity will understand how evolution is the fundamental concept of biology, be able to identify and describe eukaryotic organisms, and develop the knowledge, skills and language needed for future courses in biology. Following the course, the student should expect to:

\begin{itemize}
	\item Explain the theory of evolution by natural selection and understand when evolution can occur
	\item Demonstrate understanding of the roll of Hardy-Weinberg Theory in explaining population genetics
	\item Describe modes of speciation and radiation of new species
	\item Understand what selection pressures (agents) are and their role in adaptation
	\item Demonstrate an understanding of evolution as the foundation of biology
	\item Describe the unique evolutionary pathways and characteristics of each Kingdom
	\item Demonstrate knowledge of animals, plants, fungi, and protists
	\item Compare and contrast various taxonomic groups
	\item Recognize the diversity of adaptations organisms have for solving life's problems
	\item Know major characteristics of taxa and how these are used in classification and phylogenies
	\item Be familiar with general laboratory investigation techniques (i.e., microscope use, hypothesis and experiment development, basic analysis and dissection techniques, etc.)
\end{itemize}

\subsection{Required Textbooks \& Lab Material}

\begin{itemize}
	\item Urry, Cain, Wasserman, Minorsky, and Reece. 2016. \emph{Campbell Biology}. 11th Edition. Pearson Education.
	\item MasteringBiology course id: BIO122ARCHMILLER2020 -- Make sure you purchase the textbook with the access codes to MasteringBiology. Other options to purchase MasteringBiology are available online when you go to complete the first assignment. 
	\item Lab packet available for purchase at Bookstore. Keep lecture notes and lab material in a 3-ring binder as your class notebook. 
	\item Colored pencils
	\item Strete and Vodopich. 2011. \emph{Photo Atlas for General Biology}. McGraw-Hill (*Strongly recommended)
	\item SimUText Darwinian Snails (instructions for purchase/installation will be posted on Moodle)
	\item Victoria E.\ McMillan. 2012. \emph{Writing Papers in the Biological Sciences}.
\end{itemize}

\section{Attendance Policy}

Regular attendance and participation in class is critical to your success in this class and at Concordia College. Because any absence, excused or unexcused, detracts from the learning experience, \textbf{you are expected to attend all classes}. Although attendance is not formally included in your course grade, absences will be reflected in exam and lab practical scores. The material can be very challenging and there is no substitute for attending class. 

%The nature of this integrated class (integrated lecture and lab with long class periods)I also value the educational experience afforded by student participation in co-curricular activities; however, you are responsible for notifying me of scheduled absences (e.g., co-curricular activities) at the beginning of the semester, or as soon as that information is available (but no less than 24 hours in advance). 



If you know you will be gone for a school sanctioned event, please let me know ahead of time. If you are ill, please email me as soon as possible. \textbf{No matter the nature of your absence, you are still responsible for obtaining and understanding the material you missed.} If you do need to miss a class period, you may ask whether you could attend Dr.\ ArchMiller's other class section. You may also use the open lab time on Friday to review and make-up the laboratory portion of class with TAs; however, you must still obtain class notes from someone in your section.  

If you miss an exam for an excused absence (e.g. documented illness, travel for school sports or music) you must make arrangements with me to make up the exam in a time reasonable for the absence. The format of the make-up exam may be different than the one given in class (e.g. combination of written and oral exam). In-class lab and lecture activities cannot be made up.

If absences become what I determine to be excessive (from 10-15\% of classes, without valid college-recognized excuses), points will be deducted from your final percentage. In extreme cases (4 unexcused absences), I may assign a failing grade. 

\section{Accommodations for Students with Disabilities}

In accordance with the Americans with Disabilities Act, Concordia College and your instructor are committed to making reasonable accommodations to assist individuals with documented disabilities to reach their academic potential. Such disabilities include, but are not limited to, learning or psychological disabilities, or impairments to health, hearing, sight, or mobility. If you believe you require accommodations for a disability that may impact your performance in this course, you must schedule an appointment with Disability Services to determine eligibility. Students are then responsible for giving instructors a letter from Disability Services indicating the type of accommodation to be provided; please note that accommodations will not be retroactive. The Disability Services office is in Old Main 109A, phone 218-299-3514; https://cobbernet.cord.edu/directories/offices-services/counseling-center-disability-services/

\section{Respect for Diversity}

It is my intent that students from diverse backgrounds and perspectives be well-served by this course, and that the diversity that students bring to this class be viewed as a resource. Please let me know ways to improve the effectiveness of the course for you, personally, or for other students or student groups. As a student in this class, you are required to treat other members of the class with respect and kindness. Disrespectful, rude, or exclusive behavior will not be tolerated.

\end{fullwidth}

\section{Academic Integrity (from Student Handbook)}

\marginnote{Concordia College has university-wide policies about academic integrity, and all students are responsible for being familiar with and adhering to them. These policies are in place to protect students, first and foremost. \textbf{My role as instructor is to teach each of my students how to become responsible scholars.} As a student at Concordia College and as a student in this class, you are expected to fully and properly acknowledge the work of others. Every instance of plagiarism will be reported, as per the policies of the college, but please do not hesitate to ask me in advance if you think something might be questionable or if you are unsure about what is considered to be plagiarism. I am happy to help, as long as you inquire in advance! }

``The Concordia community expects all of our members to act with integrity--to act with honesty, uprightness and sincerity. Every member of our academic community is charged with the responsibility of encouraging and maintaining an environment of academic integrity.

``Academic misconduct is defined as any activity that comprises the academic integrity of the college or undermines the educational process. Academic misconduct includes but is not limited to:

\begin{itemize}
	\item cheating: using a resource other than one's own work to answer questions;
	\item plagiarism: misrepresenting another's ideas as one's own or not giving credit to the creator of a work;
	\item falsification: submitting falsified or fabricated information;
	\item facilitating others' violations: knowingly permitting or facilitating the dishonesty of others;
	\item impeding: placing barriers in the way of others' academic pursuits''
\end{itemize}

\begin{fullwidth}

\subsection{Biology Department policy on use of electronic devices (phones, smart watches, laptops, tablets, etc.)}

Faculty in the Biology Department work to make the classroom and laboratory a space conducive to student learning. We encourage writing notes by hand because it is an effective learning strategy for many students. However, the Biology Department also understands the valuable role of electronic devices in learning and scholarship. Thus, the Biology Department policy on the use of these devices in the classroom is as follows:


\begin{enumerate}
\item Electronic devices used during class time should be limited to appropriate class-related activities as outlined by the instructor. We reserve the right to check devices at any time and to ask you to put them away or leave if we see you using them inappropriately. Please reduce distractions to yourself and your fellow classmates.
\item All electronic devices must be set to silent during scheduled classroom and laboratory sessions. Tones and vibrations are distracting.
\item Only approved electronic devices (such as non-programmable calculators) may be available or used during examination periods. We expect that all non-approved electronic devices will be turned off and stored away from the exam areas.
\item Sharing calculators during exams is not allowed without permission. 
\item Cheating in any form, including through use of an electronic device, will not be tolerated. See the academic integrity policy for more information.
\end{enumerate}

Inappropriate or distracting use of electronic devices in the classroom may adversely affect your course grade. 

\end{fullwidth}

\subsection{Grades}

%\begin{table}
\begin{tabular}{l l l r}
Category &  Date & Partial \%  \\
\hline
 Exam 1 w/lab & Feb.\ 3 & 15 \\							% change date for each unit
 Exam 2 w/lab & Feb.\ 26 & 15 \\							% change date  for each unit
 Exam 3 w/lab & Mar.\ 25 & 15 \\						% change date for each unit
 Lab Exam 3 & April 22 & 5 \\
 Lecture Exam 3 & May 1 & 15 \\ 
 Notebook \& Pre-labs & various & 15 \\							% change date for each unit
Participation \& &  various  & 20 \\
\, Assignments \\
\hline
Total & &   100\%
\end{tabular}
%\end{table}


\begin{margintable}
%Final grades will be based on the following scale:\\
\begin{tabular}{rl}
Percentage & Grade \\
\hline 
$\ge94$ & A \\
90-93.9 & A- \\
87-89.9 & B+ \\
83-86.9 & B \\
80-82.9 & B- \\
77-79.9 & C+ \\
73-76.9 & C \\
70-72.9 & C- \\
67-69.9 & D+ \\
60-66.9 & D \\
$<60$ & F \\
\hline
\end{tabular}
\end{margintable}




\begin{fullwidth}


%This course will be a combination of quizzes, brief lectures, questions, discussion, activities, and laboratory. In order to learn the material, students must actively participate in learning. There is a limited amount of time; therefore you must come prepared to dive into the material. 


\newthought{{Exams}} will be of variable format, including---but not limited to---multiple choice, true/false, matching, short answer, brief essays, and lab practicals (Exams 2, 3, and 4 only). All exams will be cumulative by nature; however $\sim$25\% of the final lecture exam will be designated for cumulative material. 

\newthought{{Participation \& Assignments}} include in-class participation and exercises, MasteringBiology assignments, and occasional homework assignments. MasteringBiology will be used to provide reading comprehension assignments that will be due on most Mondays by 11:59pm. 

Bring colored pencils to class. You are also encouraged to bring a laptop or tablet to class each day for activities. Laptops will be made available if you need one. For lab safety reasons, coats, backpacks, and other gear should be placed in the cubbies near the front of the lab. 

It is essential that you prepare for each class in advance and review each lab soon after its completion. Past students have found success using various study techniques such as flash cards, a student-run Biology 122 Facebook page, reviewing on Fridays in open lab, and group study sessions. 

\newthought{Lecture and Lab Notebook} will be turned in each Friday for grading. You are expected to date each lab assignment, have complete notes, and detailed, labeled drawings (color is better).  {Pre-labs} will be due at the beginning of most class periods. Please see schedule and Moodle for specific dates.

\newthought{Extra Credit} may be earned by attending special lectures scheduled throughout the semester.  You will be required to hand in a summary and your reaction of the lecture to receive the extra credit (1 page, 1.5 spacing, 1 inch margins). Known opportunities include:

\begin{itemize}
\item Martin Luther King, Jr. Day, January 20. For extra credit, you must attend one of the concurrent sessions and write a reflection paper that discusses how diversity enriches the human experience. 
\item Celebration of Student Scholarship, April 15.  For extra credit, you must attend either 5 poster presentations or 3 oral presentations, or some combination of the two.  Your reflection paper must include complete title, authors and session for each presentation, a 1 paragraph summary for each presentation (oral or poster), and your reflections on the presentation.  	
%\item Gender Matters Expo, February 14th 4-6pm. Write a reflection paper that discusses how diversity enriches the human experience.
%\item Dr.\ ArchMiller's Centennial Scholars Lecture, Tuesday February 11 at 7pm in Jones A/B. write a reflection paper that discusses how this reproducibility improves science.
\end{itemize}









\newpage
\subsection{Course Schedule (version dated 1/8/2020)}
%
%\begin{itemize}
%	\item Lecture: Lecture Topic
%	\item Lab: Lab Topic
%	%\item MB: MasteringBiology Assignments
%	\item Campbell Biology: Campbell Biology Chapters
%	%\item OH: Dr.~ArchMiller's office hours, ISC 222 (MW 10:30-11:30am; TR 2:00-3:00pm)
%\end{itemize}

\textbf{Topic:} You are expected to read the Campbell Biology chapters prior to coming to lecture.  There will be MasteringBiology assignments due on Fridays at 11:59pm throughout the semester that will test you on reading comprehension. 

\textbf{Activity:} You are expected to print and read-through the required lab material before class. Pre-labs will be due at the beginning of class. Keep all lab materials (handouts, pre-labs, whiteboard notes, etc) in a 3-ring-binder. Your notes and drawings will be assessed each Friday.

\textbf{Open Lab Hours:} will be held in ISC 256 every Friday from 9:15-10:30am and 12:00-4:30pm.

%\newthought{Important dates:} 

%Martin Luther King, Jr.\ Day: January 21, 2019

%Last day to drop: March 27, 2019

%Registration: Week of March 19, 2019

%COSS: April 10, 2019

  \setlength{\calwidth}{6.5in}
  \setlength{\calboxdepth}{0.3in}
  \begin{calendar}{1/6/20}{17}

 \calday[Monday]{\classday} % Monday
  \skipday%[Tuesday]{\classday} % Wednesday
 \calday[Wednesday]{\classday}
  \skipday%[Thursday]{\classday} % Thursday (unnumbered)
 \calday[Friday]{\noclassday} % Friday
    \skipday
    \skipday % weekend (no class)
   % \skipday


%%%%%%%%%%%%%%%%%%%%%%%%%%%%%%%%%%%%%%%%%%%%%%%%%%%%%%%% Week 1
\caltext{1/10/20}{Watch Judgement Day video. 
\begin{itemize}
\item 8-10am 
\item 10:30-12:30
\item 12:30-2:30
\item 2:30-4:30
\end{itemize}}
\caltext{1/11/19}{
	\begin{itemize}
	\item Complete: ``Judgement Day''
	\end{itemize}
}

%\caltext{1/14/19}{\textbf{First Day of Class}} % Lecture/Lab 1
\caltext{1/13/20}{Topic: Introduction to and Evidence for Evolution}
\caltext{1/13/20}{Campbell Biology: Ch 22\dotfill}
\caltext{1/13/20}{Activity: Darwinian Snails}
\caltext{1/13/20}{
	\begin{itemize}
	\item Bring laptop/tablet
	\item Print/Bring: ``Seely Article''  
	\end{itemize}
}

\caltext{1/15/20}{Topic: Microevolution}% Lecture/Lab 2
\caltext{1/15/20}{Campbell Biology: Ch 23\dotfill}
\caltext{1/15/20}{Activity: Darwinian Snails }
\caltext{1/15/20}{
	\begin{itemize}
	\item Complete Darwinian Snails Tutorial prior to class
	\item Bring laptop/tablet
	\item Complete Darwinian Snails Experimental Planning
	\end{itemize}
}

\caltext{1/17/20}{Topic: Natural Selection} % Lecture/Lab 1
\caltext{1/17/20}{Campbell Biology: Ch 23\dotfill}
\caltext{1/17/20}{Notebook Check:
	\begin{itemize}
	\item ``Judgement Day''
	\item ``DS - Comparing Literature''
	\item ``DS - Experimental Planning''
	\end{itemize}
}

%%%%%%%%%%%%%%%%%%%%%%%%%%%%%%%%%%%%%%%%%%%%%%%%%%%%%%%% Week 2
\caltext{1/20/20}{MLK Jr.\ Day} % Special Note 



\caltext{1/22/20}{Topic: Hardy-Weinberg Equilibrium} % Lecture/Lab 2
\caltext{1/22/20}{Campbell Biology: Ch 23\dotfill}
\caltext{1/22/20}{Activity: Hardy-Weinberg}
\caltext{1/22/20}{%Lab Material:
	\begin{itemize}
	\item Read ``Sickle Cell Anemia''
	\item Read ``Hardy-Weinberg''
	\end{itemize}
}


\caltext{1/24/20}{Topic: Speciation and Macroevolution} % Lecture/Lab 1
\caltext{1/24/20}{Campbell Biology: Ch 24\dotfill}
\caltext{1/24/20}{Notebook Check:
	\begin{itemize}
	\item ``Hardy-Weinberg Graphs''
	\item ``Hardy-Weinberg Practice Problems''
	\end{itemize}
	}
\caltext{1/24/20}{MasteringBio Ch 22/23 Due}

%%%%%%%%%%%%%%%%%%%%%%%%%%%%%%%%%%%%%%%%%%%%%%%%%%%%%%%% Week 3
\caltext{1/27/20}{Topic: Evolutionary Relationships} % Lecture/Lab 1
\caltext{1/27/20}{Campbell Biology: Ch 26\dotfill}
\caltext{1/27/20}{Activity: Fruity Phylogeny}
\caltext{1/27/20}{%Lab Material:
	\begin{itemize}
	\item Pre-lab due: Fruity Phylogeny
	\end{itemize}
}




\caltext{1/29/20}{Topic: Molecular Relationships} % Lecture/Lab 1
\caltext{1/29/20}{Campbell Biology: Ch 26\dotfill}
\caltext{1/29/20}{Activity: Molecular Relatedness}
\caltext{1/29/20}{%Lab Material:
	\begin{itemize}
	\item Bring laptop
	\item Pre-lab due: Molecular Relatedness
	\end{itemize}
}


\caltext{1/31/20}{Topic: Molecular Relationships} % Lecture/Lab 1
\caltext{1/31/20}{Campbell Biology: Ch 26\dotfill}
\caltext{1/31/20}{Notebook Check:
	\begin{itemize}
	\item ``Fruity Phylogeny''
	\item ``Molecular Relatedness''
	\item ``Hardy-Weinberg Practice Problems''
	\end{itemize}
	}
\caltext{1/31/20}{MasteringBio Ch 24/26 Due}


%%%%%%%%%%%%%%%%%%%%%%%%%%%%%%%%%%%%%%%%%%%%%%%%%%%%%%%% Week 4

\caltext{2/3/20}{\textbf{Exam 1}} % Lecture/Lab 2

	
\caltext{2/5/20}{Topic: Intro to Animals} % Lecture/Lab 1
\caltext{2/5/20}{Campbell Biology: Ch 25 \& 32 }
\caltext{2/5/20}{Activity: Microscope, Porifera and Cnidaria}
\caltext{2/5/20}{%Lab Material:
	\begin{itemize}
	\item Pre-lab Due: Microscope, Porifera, Cnidaria
	\end{itemize}
}

\caltext{2/7/20}{Topic: Porifera \& Cnidaria} % Lecture/Lab 1
\caltext{2/7/20}{Campbell Biology: 32 }
\caltext{2/7/20}{Notebook Check:
	\begin{itemize}
	\item ``Porifera and Cnidaria''
		\end{itemize}
}

\caltext{2/7/20}{Darwinian Snails Paper Due on Moodle}



%%%%%%%%%%%%%%%%%%%%%%%%%%%%%%%%%%%%%%%%%%%%%%%%%%%%%%%% Week 5
\caltext{2/10/20}{Topic:  Lophotrochozoa: Platyhelminthes, Syndermata, Ectoprocta} % Lecture/Lab 2
\caltext{2/10/20}{Campbell Biology: Ch 33\dotfill}
\caltext{2/10/20}{Activity: Platyhelminthes}
\caltext{2/10/20}{%Lab Material:
	\begin{itemize}
	\item Pre-lab Due: Platyhelminthes
	\end{itemize}
}

\caltext{2/12/20}{Topic:  Lophotrochozoa: Brachiopoda, Mollusca} % Lecture/Lab 2
\caltext{2/12/20}{Campbell Biology: Ch 33\dotfill}
\caltext{2/12/20}{Activity: Mollusca}
\caltext{2/12/20}{%Lab Material:
	\begin{itemize}
	\item Pre-lab Due: Mullusca
	\end{itemize}
}

\caltext{2/14/20}{Topic: Lophotrochozoa } % Lecture/Lab 1
\caltext{2/14/20}{Campbell Biology: Ch 33\dotfill}
\caltext{2/14/20}{Notebook Check:
	\begin{itemize}
	\item ``Platyhelminthes''
	\item ``Mollusca''
	\end{itemize}
	}
\caltext{2/14/20}{MasteringBio Ch 25/32 Due}

%%%%%%%%%%%%%%%%%%%%%%%%%%%%%%%%%%%%%%%%%%%%%%%%%%%%%%%% Week 6


\caltext{2/17/20}{Topic: Annelida, Ecdysozoa: Nematoda} % Lecture/Lab 2
\caltext{2/17/20}{Campbell Biology: Ch 33\dotfill}
\caltext{2/17/20}{Activity: Annelida \& Nematoda}
\caltext{2/17/20}{%Lab Material:
	\begin{itemize}
	\item Pre-labs Due: Annelida \& Nematoda
	\item Finish: ``Worm comparison''
	\end{itemize}
}

\caltext{2/19/20}{Topic: Ecdysozoa: Arthropoda} % Lecture/Lab 2
\caltext{2/19/20}{Campbell Biology: Ch 33\dotfill}
\caltext{2/19/20}{Activity: Arthropoda }
\caltext{2/19/20}{%Lab Material:
	\begin{itemize}
	\item Pre-lab Due: Arthropoda
	\end{itemize}
}

\caltext{2/21/20}{Topic: Ecdysozoa: Arthropoda} % Lecture/Lab 1
\caltext{2/21/20}{Campbell Biology: Ch 33\dotfill}
\caltext{2/21/20}{Notebook Check:
	\begin{itemize}
	\item ``Nematoda'' 
	\item ``Annelida''
	\item ``Arthropoda''
	\item ``Worm Comparison''
	\end{itemize}
	}
\caltext{2/21/20}{MasteringBio Ch 33 Due}

%%%%%%%%%%%%%%%%%%%%%%%%%%%%%%%%%%%%%%%%%%%%%%%%%%%%%%%% Week 7

\caltext{2/24/20}{Topic: Ecdysozoa: Echinodermata} % Lecture/Lab 2
\caltext{2/24/20}{Campbell Biology: Ch 33\dotfill}
\caltext{2/24/20}{Activity: Echinodermata}
\caltext{2/24/20}{%Lab Material:
	\begin{itemize}
	\item Pre-lab Due: Echinodermata
	\end{itemize}
}


\caltext{2/26/20}{\textbf{Exam 2}} % Lecture/Lab 1
\caltext{2/26/20}{Notebook Check:
	\begin{itemize}
	\item ``Echinodermata''
	\end{itemize}
	}

\caltext{2/28/20}{Topic: Intro Fungi} % Lecture/Lab 2
\caltext{2/28/20}{Campbell Biology: Ch 27\dotfill}


%%%%%%%%%%%%%%%%%%%%%%%%%%%%%%%%%%%%%%%%%%%%%%%%%%%%%%%% Week 8
\caltext{3/4/20}{Break: No class} %Monday 
\caltext{3/5/20}{Break: No class} %Tuesday 
\caltext{3/6/20}{Break: No class} %Wednesday 
\caltext{3/7/20}{Break: No class} %Thurs 
\caltext{3/2/20}{Break: No class}

%\caltext{2/26/18}{OH:10:30-11:30am} %Monday OH
%\caltext{2/27/18}{OH: 2pm-3pm} %Tuesday OH
%\caltext{2/28/18}{OH:10:30-11:30am} %Wednesday OH
%\caltext{3/1/18}{OH: 2pm-3pm} %Thurs OH

%%%%%%%%%%%%%%%%%%%%%%%%%%%%%%%%%%%%%%%%%%%%%%%%%%%%%%%% Week 9
\caltext{3/9/20}{Topic: Fungi} % Lecture/Lab 2
\caltext{3/9/20}{Campbell Biology: Ch 31\dotfill}
\caltext{3/9/20}{Activity: Fungi}
\caltext{3/9/20}{%Lab Material:
	\begin{itemize}
	\item Pre-lab Due: Fungi
	\end{itemize}
}

\caltext{3/11/20}{Topic: Fungi} % Lecture/Lab 2
\caltext{3/11/20}{Campbell Biology: Ch 31\dotfill}
\caltext{3/11/20}{Activity: Fungi}
\caltext{3/11/20}{%Lab Material:
	\begin{itemize}
	\item  Continue Fungi
	\end{itemize}
}

\caltext{3/13/20}{Topic: Fungi} % Lecture/Lab 1
\caltext{3/13/20}{Campbell Biology: Ch 31\dotfill}

\caltext{3/13/20}{MasteringBio Ch 27/31 Due}

%%%%%%%%%%%%%%%%%%%%%%%%%%%%%%%%%%%%%%%%%%%%%%%%%%%%%%%% Week 10
\caltext{3/16/20}{Topic: Intro Protists} % Lecture/Lab 2
\caltext{3/16/20}{Campbell Biology: Ch 28\dotfill}
\caltext{3/16/20}{Activity: Protists}
\caltext{3/16/20}{%Lab Material:
	\begin{itemize}
	\item Pre-lab Due: Protists
	\end{itemize}
}

\caltext{3/18/20}{Topic: Protists} % Lecture/Lab 2
\caltext{3/18/20}{Campbell Biology: Ch 28\dotfill}
\caltext{3/18/20}{Activity: Protists
	\begin{itemize}
	\item Continue ``Protists''
	\end{itemize}
	}

\caltext{3/20/20}{Topic: Protists} % Lecture/Lab 1
\caltext{3/20/20}{Campbell Biology: Ch 28\dotfill}
\caltext{3/20/20}{Notebook Check:
	\begin{itemize}
	\item ``Fungi''
	\end{itemize}
}
\caltext{3/20/20}{MasteringBio Ch 28 Due}

%%%%%%%%%%%%%%%%%%%%%%%%%%%%%%%%%%%%%%%%%%%%%%%%%%%%%%%% Week 11
\caltext{3/23/20}{Topic: Protists} % Lecture/Lab 2
%\caltext{3/25/20}{Campbell Biology: Ch 33\dotfill}
%\caltext{3/25/20}{Activity: None}
\caltext{3/23/20}{Notebook Check:
	\begin{itemize}
	\item ``Protists''
	\end{itemize}
	}

%\caltext{3/27/20}{Topic: Review} % Lecture/Lab 2
%\caltext{3/25/20}{Campbell Biology: Ch 33\dotfill}
%\caltext{3/25/20}{Activity: None}
\caltext{3/25/20}{\textbf{Exam 3}} % Lecture/Lab 1

\caltext{3/27/20}{Topic: Intro to Plant Kingdom} % Lecture/Lab 1
\caltext{3/27/20}{Campbell Biology: Ch 29\dotfill}
%\caltext{3/29/20}{Assignment given: Tree Walk}


%%%%%%%%%%%%%%%%%%%%%%%%%%%%%%%%%%%%%%%%%%%%%%%%%%%%%%%% Week 12


\caltext{3/30/20}{Topic: Bryophytes} % Lecture/Lab 2
\caltext{3/30/20}{Campbell Biology: Ch 29\dotfill}
\caltext{3/30/20}{Activity: Bryophytes}
\caltext{3/30/20}{%Lab Material:
	\begin{itemize}
	\item Pre-lab Due: Bryophytes
	\end{itemize}
}

\caltext{4/1/20}{Topic: Monilophyta} % Lecture/Lab 2
\caltext{4/1/20}{Campbell Biology: Ch 29\dotfill}
\caltext{4/1/20}{Activity: Monilophyta}
\caltext{4/1/20}{%Lab Material:
	\begin{itemize}
	\item Pre-lab Due: Monilophyta
	\end{itemize}
}

\caltext{4/3/20}{Topic: Monilophyta} % Lecture/Lab 1
\caltext{4/3/20}{Campbell Biology: Ch 29\dotfill}
\caltext{4/3/20}{Notebook Check:
	\begin{itemize}
	\item ``Bryophytes''
	\end{itemize}
	}
\caltext{4/3/20}{MasteringBio Ch 29 Due}


%%%%%%%%%%%%%%%%%%%%%%%%%%%%%%%%%%%%%%%%%%%%%%%%%%%%%%%% Week 13
\caltext{4/6/20}{Topic: Gymnosperms} % Lecture/Lab 2
\caltext{4/6/20}{Campbell Biology: Ch 30\dotfill}
\caltext{4/6/20}{Activity: Gymnosperms}
\caltext{4/6/20}{%Lab Material:
	\begin{itemize}
	\item Pre-lab Due: Gymnosperms
	\end{itemize}
}



\caltext{4/8/20}{Topic: Angiosperm Reproduction} % Lecture/Lab 2
\caltext{4/8/20}{Campbell Biology: Ch 38\dotfill}
\caltext{4/8/20}{Activity: Angiosperm Reproduction}
\caltext{4/8/20}{%Lab Material:
	\begin{itemize}
	\item Pre-lab Due: Angiosperm Reproduction
	\end{itemize}
}

\caltext{4/10/20}{Easter: No Class} % Special Note

%\caltext{4/10/20}{Topic: Angiosperm} % Lecture/Lab 1
%\caltext{4/10/20}{Campbell Biology: Ch 35\dotfill}



%%%%%%%%%%%%%%%%%%%%%%%%%%%%%%%%%%%%%%%%%%%%%%%%%%%%%%%% Week 14

\caltext{4/13/20}{Easter: No Class} % Special Note

\caltext{4/15/20}{COSS} % Special Note


\caltext{4/17/20}{Topic: Angiosperm Reproduction} % Lecture/Lab 2
\caltext{4/17/20}{Campbell Biology: Ch 38\dotfill}
\caltext{4/17/20}{Notebook Check:
	\begin{itemize}
	\item ``Gymnosperms''
	\item ``Angiosperm Reproduction''
	\end{itemize}
	}
\caltext{4/17/20}{MasteringBio Ch 30 Due}

%%%%%%%%%%%%%%%%%%%%%%%%%%%%%%%%%%%%%%%%%%%%%%%%%%%%%%%% Week 15

\caltext{4/20/20}{Topic: Angiosperm Structure} % Lecture/Lab 2
\caltext{4/20/20}{Campbell Biology: Ch 35\dotfill}
\caltext{4/20/20}{Activity: Angiosperm Structure}
\caltext{4/20/20}{%Lab Material:
	\begin{itemize}
	\item Pre-lab Due: Angiosperm Structure
	\end{itemize}
}

\caltext{4/22/20}{\textbf{Lab Exam 3}}
%\caltext{4/22/20}{Topic: Angiosperm Structure} % Lecture/Lab 2
%\caltext{4/22/20}{Campbell Biology: Ch 35\dotfill}
\caltext{4/22/20}{Notebook Check:}
\caltext{4/22/20}{%Lab Material:
	\begin{itemize}
	\item ``Angiosperm Structure''
	\end{itemize}
}



\caltext{4/27/20}{Tree Walk due on Moodle}
\caltext{4/24/20}{MasteringBio Ch 35/38 Due}



%%%%%%%%%%%%%%%%%%%%%%%%%%%%%%%%%%%%%%%%%%%%%%%%%%%%%%%% Week 16

\caltext{5/1/20}{\textbf{Lecture Exam 3} 8:30-10:30am} % Lecture/Lab 2

										% change for each unit

  \end{calendar}

%\newthought{Final Exams:} Lecture Exam 3 on April 29, 2019 8:30-10:30am

\end{fullwidth}




\end{document}                              