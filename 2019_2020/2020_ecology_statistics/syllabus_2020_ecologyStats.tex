\documentclass{tufte-handout}
\usepackage[letterpaper]{geometry}
\usepackage{fontspec}
\usepackage{termcal}
\usepackage{graphicx}


\geometry{
  left=1in, % left margin
  bottom=1in,
  top=1in,
  textwidth=25pc, % main text block
  marginparsep=2pc, % gutter between main text block and margin notes
  marginparwidth=12pc % width of margin notes
}

\makeatletter
\providecommand\tuftedate{}
\@ifpackageloaded{termcal}{%
  \renewcommand{\date}[1]{%
    \gdef\@date{#1}%
    \begingroup%
    % TODO store contents of \thanks command
    \renewcommand{\thanks}[1]{}% swallow \thanks contents
    \protected@xdef\tuftedate{#1}%
    \endgroup%
  }{%
    % Do nothing else, there's no need to redefine \date
  }
}
\makeatother
\defaultfontfeatures{Mapping=tex-text}

\renewcommand{\allcapsspacing}[1]{{\addfontfeature{LetterSpace=10.0}#1}}
\renewcommand{\smallcapsspacing}[1]{{\addfontfeature{LetterSpace=2.0}#1}}
\renewcommand{\textsc}[1]{\smallcapsspacing{\textsmallcaps{#1}}}
\renewcommand{\smallcaps}[1]{\smallcapsspacing{\scshape\MakeTextLowercase{#1}}}

\renewcommand{\calprintclass}{}

\title{2020 Syllabus for BIOL480: Independent Study: Ecology Statistics}										% change each year
\author{Tuesdays 4-5pm; Thursdays 3-5pm}										% change per section
\date{308A ISC}

\begin{document}
\maketitle

Instructor: Dr.~Althea A.~ArchMiller\marginnote{The schedules and policies associated with this course may be subject to revision or change as a consequence of changing circumstances or events. Reasonable notification will be provided to students prior to any major changes in course policies or procedures.}\\
Office: 220 Integrated Science Center\\
218.299.3793 (office) / 218.556.8053 (cell)\\
Email: aarchmil@cord.edu\\
Twitter: @aaarchmiller\\
Office Hours: MW 10:30am-12:00pm; T 8-9:30am


\begin{fullwidth}

\section{Course Description \& Goals}

This course is designed to provide students with hands-on skills needed to design experiments and analyze ecological data efficiently using appropriate statistical models and program R. 

%\subsection{Course Goals}

The primary objective of this course is to provide hands-on experience analyzing ecological data. 

\newthought{Learning outcomes} are to:

\begin{enumerate}
	\item Understand basic approaches and assumptions underlying common statistical techniques
	\item Design experiments to test hypotheses
	\item Follow and justify appropriate model selection techniques
	\item Use analysis programs such as R
	\item Read and synthesize relevant and contempory literature
	\item Adopt and follow a reproducible workflow for statistical analysis
	\item Overcome challenges inherent in scientific research and analysis
\end{enumerate}


\end{fullwidth}

%\subsection{Grades}

\newthought{Grades} in this class will be more closely tied to your growth and development as a statistician in the following areas than they will be towards specific tasks or competencies. 

\begin{margintable}
\begin{tabular}{rl}
Percentage & Grade \\
\hline 
$\ge93$ & A \\
90-92.9 & A- \\
87-89.9 & B+ \\
83-86.9 & B \\
80-82.9 & B- \\
77-79.9 & C+ \\
73-76.9 & C \\
70-72.9 & C- \\
67-69.9 & D+ \\
60-66.9 & D \\
$<60$ & F \\
\hline
\end{tabular}
\end{margintable}

The format of this class will be a mixture of introductory lectures, independent reading assignments, and analysis in program R. Each topic will have an associated assignment, which will need to be completed in R and converted into an html. All work must be updated on the GitHub repository by the assigned date and time. Specific due dates and assignments will be posted on the white board in ISC 308A.

By the end of the semester, I expect you to have accomplished an understanding of the following statistical topics:

\begin{itemize}
	\item Statistical distributions
	\item Experimental design
	\item t-tests and ANOVA
	\item Linear regression
	\item ANCOVA and mixed models
	\item Model selection
\end{itemize}



\begin{fullwidth}

\subsection{Attendance Policy}

Regular attendance and participation in class is critical to your success at Concordia College. Because any absence, excused or unexcused, detracts from the learning experience, you are expected to be present during the scheduled course time. However, I also understand that you are an advanced learner capable of balancing your personal schedule, coursework, and research. As such, I trust you to make up the requisite  hours if you are absent during scheduled course times. 


\subsection{Accommodations for Students with Disabilities}

In accordance with the Americans with Disabilities Act, Concordia College and your instructor are committed to making reasonable accommodations to assist individuals with documented disabilities to reach their academic potential. Such disabilities include, but are not limited to, learning or psychological disabilities, or impairments to health, hearing, sight, or mobility. If you believe you require accommodations for a disability that may impact your performance in this course, you must schedule an appointment with Disability Services to determine eligibility. Students are then responsible for giving instructors a letter from Disability Services indicating the type of accommodation to be provided; please note that accommodations will not be retroactive. The Disability Services office is in Old Main 109A, phone 218-299-3514; cobbernet.cord.edu/directories/offices-services/counseling-center-and-disability-services/disability/


\subsection{Respect for Diversity}

It is my intent that students from diverse backgrounds and perspectives be well-served by this course, and that the diversity that students bring to this class be viewed as a resource. Please let me know ways to improve the effectiveness of the course for you, personally, or for other students or student groups. As a student in this class, you are required to treat other members of the class with respect and kindness. Disrespectful, rude, or exclusive behavior will not be tolerated.


\end{fullwidth}

\section{Academic Integrity (from Student Handbook)}

\marginnote{Concordia College has university-wide policies about academic integrity, and all students are responsible for being familiar with and adhering to them. These policies are in place to protect students, first and foremost. \textbf{My role as instructor is to teach each of my students how to become responsible scholars.} As a student at Concordia College and as a student in this class, you are expected to fully and properly acknowledge the work of others. Every instance of plagiarism will be reported, as per the policies of the college, but please do not hesitate to ask me for clarification in advance.}

``The Concordia community expects all of our members to act with integrity--to act with honesty, uprightness and sincerity. Every member of our academic community is charged with the responsibility of encouraging and maintaining an environment of academic integrity.

``Academic misconduct is defined as any activity that comprises the academic integrity of the college or undermines the educational process. Academic misconduct includes but is not limited to cheating, plagiarism, falsification, facilitation, or impeding.

%\begin{itemize}
%	\item cheating: using a resource other than one's own work to answer questions;
%	\item plagiarism: misrepresenting another's ideas as one's own or not giving credit to the creator of a work;
%	\item falsification: submitting falsified or fabricated information;
%	\item facilitating others' violations: knowingly permitting or facilitating the dishonesty of others;
%	\item impeding: placing barriers in the way of others' academic pursuits''
%\end{itemize}

%\newpage






\end{document}                              